\documentclass[12pt,a4paper]{article}
\usepackage{graphicx}
\usepackage{cite}
\usepackage{latexsym}
\usepackage{fullpage}
\usepackage{gensymb}

\title{Comparing the Performance of Modified OLSR Protocols}
\author{Callan Gray, Nathan Stanley, James Udiljak}
\date{\today}

\begin{document}

\maketitle

\begin{abstract}
We compare the performance of the standard optimized link state routing (OLSR) protocol for ad-hoc wireless networking with several modified versions. The first modification involves including all one-hop neighbours of a node in that node's MPR set. Secondly, we modify OLSR to choose a random subset of the regular MPR set for each node, which is half the size of the original MPR set. Thirdly, we modify topology control (TC) messages sent by nodes to contain a random, half-length subset of the original MPR set. Finally we remove the condition that links to nodes in a node's MPR set must be bidirectional. Using ns-3, we compare the performance of these modified OLSR protocols to each other, as well as the original, unmodified protocol, for a network of 50 mobile nodes in a 1.5km by 1.5km area, each transmitting constant bit rate (CBR) traffic.
\end{abstract}

\section{The OLSR protocol}
Optimized link state routing (OLSR) is a networking protocol of the link state class. OLSR was designed to service the transmission of data across wireless networks of mobile nodes, without the use of a central router\cite{clausen2003optimized}. The first routing algorithm utilising link state routing was published in 1978 \cite{mcquillan1978arpanet}. It was developed for use in ARPANet. OLSR is based on traditional link state algorithms, but is optimised for wireless, ad-hoc networks. In link state routing protocols, every node constructs its own map of the network, represented by a graph. Nodes use this graph to independently calculate routing tables, which detail the node's best known path to each other node in the network. Only information about network topology is shared between nodes. By comparison, in distance vector routing protocols, routing tables, or parts thereof, are shared between nodes.

\subsection{The MPR Set}
Each node maintains a set of nodes which it considers multipoint relays (MPRs), termed that node's MPR set. The nodes in an MPR set are the nodes which are selected by that node to forward its broadcast messages. Packets containing information about network topology (termed topology control, or TC messages) are then only generated by nodes which have been elected as an MPR by some other node. By generating MPR sets that allow a node to broadcast to and receive from the entire network with minimal redundancy, OLSR reduces its overhead of TC messages.

\subsection{Modification 1}
The first modification to the OLSR algorithm that we test involves modifying the generation of MPR sets such that all one-hop neighbours of a node will appear in that node's MPR set, as opposed to the minimum subset of those required to allow that node reach the entire network. We expect that this will increase the overhead of TC messages throughout the network and thus reduce throughput of data.

\subsection{Modification 2}
We also modify OLSR to choose a random subset of the traditionally generated MPR set for each node, which will be used as its MPR set. Since the MPR sets are usually generated to be as small as possible without reducing the node's ability to reach other nodes in the network, we expect that this will make some links unserviceable, causing throughput to drop.

\subsection{Modification 3}
By modifying TC messages so that nodes only send half of their MPR set to other nodes, nodes will be rendered unaware of the existence of some nodes within the network. We expect that this will again render some links in the network unserviceable, again causing throughput to drop.

\subsection{Modification 4}
Finally, we modify OLSR, removing the condition that links to nodes in a node's MPR set must be bidirectional. In the standard OLSR algorithm, for a node \(A\) to be in node \(B\)'s MPR set, node \(B\) must be in node \(A\)'s MPR set. I HAVE NO IDEA WHAT THIS WILL DO.

\section{ns-3}
ns-3 is an open source network simulation framework with an extensive array of plugins. ns-3 is capable of simulating every layer of networking from physical to application. It allows for the simulation of mobile nodes communicating wirelessly through an ad-hoc network. We use ns-3 to test the performance of the standard OLSR protocol, and each of our four modifications. We use a 50 node network, with nodes confined in a \(1500\) by \(1500m\) space, moving at \(1\), \(10\), and \(20ms^{-1}\). We test the network with \(10\), \(20\), and \(30\) nodes generating constant bitrate traffic, analysing network throughput.

\section{Results}
Show pretty graphs of network performance here.

\section{Conclusion}
Summarize results and explain phenomena observed.

\bibliographystyle{plain}
\bibliography{references}

\end{document}